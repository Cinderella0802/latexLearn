\documentclass{ctexbook}
\usepackage{amsmath}
\usepackage{diagbox}

\begin{document}
	\frontmatter
	\tableofcontents
	\title{第二次行课}
	\author{黄跳}
	\maketitle
	\mainmatter
	\chapter{复习}
	\section{今天天气真好}
	\subsection{可惜不能出去玩}
	\subsubsection{心情好郁闷啊}
	
	部长大大的电脑坏了,正在重修,哈哈哈
	\section{进步}
	\section*{没有编号啦,只是借用这个格式}
	
	\section{公式}
	
	\begin{gather}
		\label{gougu}
		x^{2}+y^{2}=z^{2}
	\end{gather}
	
	\begin{gather}
	\label{jifen}
		\int_{0}^{1}x^{2}=\dfrac{1}{3}
	\end{gather}
	
	我们就可以引用公式了,比如\ref{gougu}就可以得到公式了.或者\eqref{jifen},就可以得到有括号的公式了.
	
	\section{ 脚注}
	脚注\footnote{看过来看过来看过来}
	
	\section{强调}
	我要\emph{强调}这两个字 
	
	\section{列表环境}
	\begin{itemize}
		\item 第一行
		\item 第二行
	\end{itemize} 
	
	\section*{生成带序号的列表}
	\begin{enumerate}
		\item 第一行
		\item 第二行
	\end{enumerate}
	
	\section*{嵌套}
	\begin{itemize}
		\item 第一行
		\begin{enumerate}
			\item 第一行
			\item 第二行
		\end{enumerate}
		\item 第二行
		
	\end{itemize}
	
	\section*{描述}
	\begin{description}
		\item[标签] 什么标签
	\end{description}
	
	\section*{继续嵌套}
	
	\section{对齐}
	\begin{flushleft}
		左对齐
	\end{flushleft}
	
	\begin{flushright}
		右对齐
	\end{flushright}
	
	\begin{center}
		居中
	\end{center}
	
	\section{引用}
	曾经听过一句歌词,梁静茹的可惜不是你
	\begin{quote}
		你曾说时间还很多
	\end{quote}
	
	\section{原文打印}
	\begin{verbatim}
	一二三
	四五六
	七八九
	哈哈哈 
	\end{verbatim}
	
	\section{表格}
	\begin{center}
		\begin{tabular}{|c|c|c|c|}
			\hline
			一&二&三&四\\
			\hline
			\multicolumn{4}{|c|}{我们合并为一行}\\
			\hline
			\diagbox{一}{二}& \textsl{}&&\\
			\hline
		\end{tabular}
	\end{center}
	
	\chapter{数学公式}
	行内数学模式就长这样$ a+b=c $.
	行间数学模式就长这样\[ a+b=c \]
	\section{数学公式的群组}
	$ e^i\pi=1 $
	$ e^{i\pi}=1 $
	
	\begin{equation}
	 e^{i\pi} =1
	\end{equation}
	
	\begin{equation};
	e^{i\pi}=1
	\end{equation}
	
	\section*{希腊字母}
	
	$ \alpha $
	$ \beta $
	$ \gamma $
	$ \alpha\beta\gamma $
	
	\section*{上下标}
	$ x^{2}_{1} $
	
	\section*{开方}
	$ \sqrt[3]{27} $\\$ \sqrt{16} $
	
	$ \overline{A} $
	
	
	\appendix
	\chapter{心情指数}
	\chapter{灵活程度}
	 
	 \backmatter
	 \chapter{参考文献}
	
\end{document}